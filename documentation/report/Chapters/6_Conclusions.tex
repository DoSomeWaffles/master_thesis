\chapter{Conclusions and Future Work}
\label{ch:conclusions}

\section{Objectives fullfillment}

Based on the chapter \ref{ch:results}, objectives 1, 2, 3, and 5 are fulfilled. Objective 4 is partially fulfilled. 

\section{Future Work}

We created the project for the master's thesis. We were not sure where it would lead us. After the results obtention, we can assume that we set the baseline, and other projects can use it to improve the results. A bachelor's thesis has already begun based on the baseline defined in this project. 

\subsection{Adding more microphones to the recording system}


Two microphones might not be enough for a machine learning algorithm to differentiate a sound from the left or the right. The next could be to improve the recording system by adding more microphones. This improvement would modify the dataset by adding dimensions. The current system is limited to two microphones. The system should be able to record four or more microphones. By adding channels to the audio signal, we augment the input dimension of the convolutional neural network, and we can better train it, thus improving the results.

\subsection{Labelization of more data}

Currently, the dataset only comprises a bit more than 2000 samples. By labeling more data, we can compare the results and see if the results are improving. If they improve, we can assume that the dataset needs to be way bigger for the machine learning algorithm to generalize the problem.

\subsection{Adding more classes}

The current baseline is only able to differentiate between four classes. The next step is to add more classes to the dataset. The classes could be more zones that the model would have to discern. Adding classes would also increase the problem's complexity and require more data to train the model. But it could provide a better tool to build a product we could use in real situations.

\subsection{Sound Propagation Simulation}

The sound propagation simulation was an important part of the project and is not performing well enough. The next part of improving the simulation is the implementation of a real sound propagation that simulates the time needed for the sound to reach the microphone directly into the engine and allows the use of multiple microphones at the same time while managing effects like Doppler occurring when the sound source is moving relatively to each microphone.

Not generating data for the \textit{no\_car} and the \textit{multiple\_cars} could have been an error. 

\subsection{Sound Propagation Simulation bachelor's thesis}

A thesis named \textit{SimSound3D} has started in the HEIA-FR. The goal of this thesis is to improve the sound propagation simulation. They will base their thesis on the work done in this project. At the moment, they have chosen to try to use the Unreal Engine to simulate sound propagation. The student claims to be able to record from multiple positions simultaneously. If the thesis succeeds, it could greatly improve the project.

\subsection{Advanced adversarial attack}

The current adversarial attack is a simple, inefficient attack in our situation. The next step is to improve the attack by using more advanced attacks. The patch attacks presented in [Generative Dynamic Patch Attack]\cite{li2021generative} could change a zone in the image, thus, making it easier to transform it back to an audio signal without adding lots of noise on the whole spectrogram. This process could improve the attack's efficiency and make it more realistic.

\subsection{Dataset publication}

Since we created a dataset containing more than 2000 labeled samples, we could publish it. Other researchers could use the dataset for their projects. The machine learning community could also use the dataset to improve the project results.

\section{Specification self-assessment}

At the beginning of the project, the specification was going more toward emergency vehicle detection. During the project, we modified the project's objectives to a more generic sound source localization and distance estimation. We adapted the specification to the new objectives. 

The planning was designed with three epic deadlines during the project to represent a cyclic workflow. Although the three epic deadlines did not represent three different project realization cycles, they were still useful for providing a global view during the project. The tasks defined in the planning were well followed.

The specification file was a good base to start the project and was a good way to have a global view of what we wanted to do.

\section{Personal conclusion}

It is a great opportunity to work on a project that tries to regroup every part of the IT domain, from embedded systems to machine learning. Even if this project was big and sometimes felt like multiple projects at once, it was a great experience to work on it. I learned much about the different domains and technologies, especially adversarial attacks. I also learned how to manage a project. I am happy with the results, and I am looking forward to seeing the next steps of the project.
