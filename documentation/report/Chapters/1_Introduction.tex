\chapter{Introduction}
\label{ch:introduction}
Within the framework of the research project "NPR Teleoperation," the engineers of the HEIA-FR have developed the first concept in Switzerland of a remote-controlled automated vehicle. However, teleoperation only makes sense if the vehicle is automated. There can be no teleoperation without automation (economic factors), just as there can be no automation without teleoperation (legal, technical, and social factors). ROSAS then created the Autovete (Automatisation de véhicules téléopérés) project. HEIA-FR finances them to build up vehicle automation expertise. Detecting other emergency vehicles is mandatory for a vehicle to be fully autonomous. V2V (Vehicle-to-Vehicle) communication is a solution but is not yet integrated into emergency vehicles.
So, to detect such a vehicle, two signals need to be processed: the sound of the emergency siren and the blinking lights of the vehicle. The first use case of this project focuses only on sound source distance estimation and localization. Detecting excessively noisy vehicles on the street is a simpler use case created for this project to understand if sound source estimation and localization could work for vehicles in an open environment. The goal is to measure the sound level of the passing vehicles and compare it with the legal limits. If a vehicle exceeds the limit, the system can record its license plate and report it to the authorities. This way, the system can help reduce noise pollution and improve road safety. This system requires a microphone array, a camera, and a processing unit to achieve the needed detection. The microphone array captures the sound signals from different positions and sends them to the processing unit. The processing unit applies a sound source localization algorithm to estimate the direction and distance of the sound source. The camera captures the image of the vehicle and performs license plate recognition. 

Big improvements in sound source localization with the help of machine learning are being made. They can be used to reliably localize the origin of a sound using one or more microphone arrays (multiple microphones operating in tandem). A non-negligible problem is the small number of real-world datasets with moving sources in an open environment. A solution is to create datasets in realistic sound propagation simulation and use them to augment the real-world datasets. A model can then be trained on the augmented dataset and tested for its performance in real-world data. 

Using machine learning to solve the sound source localization problem can lead to a new attack vector for the system. The system can be attacked by modifying the sound source or the sound signal. Tests to understand how the system reacts to such attacks are necessary to understand the system's robustness.

\section{Motivation}
\label{intro:motivation}

\subsection{Objectives}

\paragraph{Objective n°1 Definition of a baseline}

The first objective is to define a baseline for the project. The baseline should contain the problem statement, the project's scope, and objectives. The baseline will help us understand the project's context and the problem we are trying to solve.

\paragraph{Objective n°2 Dataset according to the baseline}

The first objective is constructing a coherent dataset with the project's baseline. The dataset contains the target variable, features, and necessary pre-processing steps. This dataset will help represent and understand the problem.

\paragraph{Objective n°3 Realising a model for better sound source localization and distance estimation}

The project uses a neural network model to detect the origin of a sound using a microphone array. The neural network uses the dataset created in objective 2 for training and can localize the sound source accurately. An evaluation of the trained neural network model in a real environment allows us to see how it performs. The model will also be evaluated to understand how it can be improved.

\paragraph{Objective n°4 Attacking the model to understand how it reacts}

The trained neural network model needs to be evaluated by testing it on data modified in some way, such as by adding or removing noise or modifying the sound source. Tests against various attacks, such as masking, time-warping, and frequency-shifting, will help us understand how it performs.

\subsection{Challenges}
\label{intro:challenges}

\paragraph{Challenge n°1: Realistic datasets}

One of the main challenges in this project is to construct realistic datasets that accurately capture the data in a real-world scenario. The lack of open-source datasets that contain moving sound sources in open environments can make this difficult. The project should use realistic simulations of sound propagation to produce suitable datasets.

\paragraph{Challenge n°2: Robust models}

Another challenge is to create a neural network model that can accurately detect the origin of a sound using a microphone array. The trained model must be accurate and robust to resist adversarial attacks. The model should localize the sound source accurately on altered data.

\paragraph{Challenge n°3: Adequate evaluation metrics}

The last challenge is to create an evaluation metric that adequately reflects the model's performance. The evaluation metric should consider the accuracy of sound source localization in a real environment and its ability to resist adversarial attacks. The metric should also capture how well the model can provide reliable results in various environments.
\section{Structure of the thesis}


\begin{itemize}
    \item \textbf{Chapter \ref{ch:background}: Background and Litterature}: This chapter provides an overview of the background knowledge necessary to understand the project, such as the theory of machine learning and sound propagation, and a brief introduction to sound source localization.
    \item \textbf{Chapter \ref{ch:methodology}: Methodology}: This chapter describes the methodology used to achieve the project's objectives, such as the dataset creation, the simulation, the neural network model, and the evaluation metrics.
    \item \textbf{Chapter \ref{ch:setup}: Realization}: This chapter describes the work done to achieve the project's objectives, such as creating a dataset, creating a simulation, building a neural network model, and evaluating it using an appropriate metric.
    \item \textbf{Chapter \ref{ch:results}: Results and Analysis}: This chapter presents the project's results, such as the performance of the neural network model and the evaluation metrics. It also analyzes the results and discusses the findings.
    \item \textbf{Chapter \ref{ch:conclusions}: Conclusion and future work}: This chapter concludes this project by discussing the main results and summarizing the key findings. 
\end{itemize}
