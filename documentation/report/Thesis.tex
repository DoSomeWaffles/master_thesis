\documentclass[oneside,a4paper]{book}

\usepackage{float}

\input{preamble}
\chapter*{\centering Abstract}
\begin{quotation}
\noindent 

Sound source localization is a well-known problem in the field of signal processing. It is used in many domains, such as robotics, surveillance, and military applications. This project aims to create a baseline to develop a system to localize a sound source in an open environment. We base our project on a neural network that takes a spectrogram as input and predicts the position of a vehicle. We train the neural network using a dataset of sounds recorded from vehicles on the street. We also use a simulation to augment the dataset. To make the model safe, we test it against adversarial attacks to understand how it behaves if someone attacks is.

\vspace{215pt}

{ centering
Supervisors:
\vspace{7.5pt}

Michael Mäder: Professor in computer science

Beat Wolf: Professor in computer science
\vspace{7.5pt}

Principals
\vspace{7.5pt}

Marc-Antoine Fénart: Professor in civil engineering

Gabriel Python: Scientific associate at Rosas

}
\end{quotation}
\clearpage
\chapter*{\centering Acknowledgements}

I want to express my gratitude to my supervisors, Michael Mäder and Beat Wolf, for the opportunity to realize this Master's thesis with their supervision and for their excellent advice during this project. I also want to thank Marc-Antoine Fénart for the help with the baseline definition and the lent material. Additionally, I want to thank Gabriel Python and every member of the Rosas team for their help, advice, and encouragement during this project. 
Finally, I would like to thank my family and friends for the support they provided during the realization of this project.

\newpage

% Acronym table

\chapter*{\centering Acronyms}
\begin{acronym}[TDMA]
    \acro{HEIA-FR}{Haute École d'Ingénierie et d'Architecture de Fribourg} 
    \acro{HES-SO}{Haute École Spécialisée de Suisse Occidentale (University of Applied Sciences and Arts Western Switzerland)}
    \acro{AI}{Artificial Intelligence}
    \acro{DNN}{Deep Neural Network}
    \acro{CNN}{Convolutional Neural Network}
    \acro{GPU}{Graphics Processing Unit}
    \acro{CPU}{Central Processing Unit}
    \acro{RAM}{Random Access Memory}
    \acro{MPEG}{Moving Picture Experts Group}
    \acro{ffmpeg}{Fast Forward MPEG}
    \acro{SSH}{Secure Shell Protocol}
    \acro{RTP}{Real-time Transport Protocol}
    \acro{SFTP}{Secure File Transfer Protocol}
    \acro{FFT}{Fast Fourier Transform}
    \acro{PCM}{Pulse-code Modulation}
    \acro{FGSM}{Fast Gradient Sign Method}
    \acro{WAV}{Waveform Audio File Format}
    \acro{MP4}{MPEG-4}
\end{acronym}




\tableofcontents



\chapter{Introduction}
\label{ch:introduction}
Within the framework of the research project "NPR Teleoperation," the engineers of the HEIA-FR have developed the first concept in Switzerland of a remote-controlled automated vehicle. However, teleoperation only makes sense if the vehicle is automated. There can be no teleoperation without automation (economic factors), just as there can be no automation without teleoperation (legal, technical, and social factors). ROSAS then created the Autovete (Automatisation de véhicules téléopérés) project. HEIA-FR finances them to build up vehicle automation expertise. Detecting other emergency vehicles is mandatory for a vehicle to be fully autonomous. V2V (Vehicle-to-Vehicle) communication is a solution but is not yet integrated into emergency vehicles.
So, to detect such a vehicle, two signals need to be processed: the sound of the emergency siren and the blinking lights of the vehicle. The first use case of this project focuses only on sound source distance estimation and localization. Detecting excessively noisy vehicles on the street is a simpler use case created for this project to understand if sound source estimation and localization could work for vehicles in an open environment. The goal is to measure the sound level of the passing vehicles and compare it with the legal limits. If a vehicle exceeds the limit, the system can record its license plate and report it to the authorities. This way, the system can help reduce noise pollution and improve road safety. This system requires a microphone array, a camera, and a processing unit to achieve the needed detection. The microphone array captures the sound signals from different positions and sends them to the processing unit. The processing unit applies a sound source localization algorithm to estimate the direction and distance of the sound source. The camera captures the image of the vehicle and performs license plate recognition. 

Big improvements in sound source localization with the help of neural networks are being made \cite{Grumiaux_2022}. They can be used to reliably localize the origin of a sound using one or more microphone arrays (multiple microphones operating in tandem). A non-negligible problem is the small number of real-world datasets with moving sources in an open environment. A solution is to create datasets in realistic sound propagation simulation and use them to augment the real-world datasets. A model can then be trained on the augmented dataset and tested for its performance in real-world data. 

Using neural networks to solve the sound source localization problem can lead to a new attack vector for the system. The system can be attacked by modifying the sound source or the sound signal. Tests to understand how the system reacts to such attacks are necessary to understand the system's robustness.

\section{Motivation}
\label{intro:motivation}

More and more cities want to fight the noise pollution in the streets. Excessively noisy vehicles are mainly causing this pollution.
The project's main objective is to help elaborate a system to detect excessively noisy vehicles on the street and report them to the authorities. The system should be able to detect the sound source's direction and distance and the vehicle's license plate. To achieve this, the system needs to localize a noisy vehicle on the street based on the sound it produces. This is important because when multiple vehicles are on a video or a picture, it is difficult to know which vehicle is noisy.

\subsection{Objectives}
The project's main objective is to help elaborate a system to localize a vehicle sound source in an open environment. The system should be able to detect the sound source's direction and distance and the vehicle's license plate.

Since we realized the specification file at the start of the project before the baseline definition, this report has updated the objectives to make them more precise and adapted to the chosen baseline. The initial objectives are available in the appendix \ref{appendix:specification}. We do a self-assessment on the specification file in the conclusion (chapter \ref{ch:conclusions}).

\paragraph{Objective n°1 Baseline}
\label{intro:objective1}

The first objective is to define a baseline for the project. The baseline should contain the problem statement and the steps to complete the baseline. It should include a list of tasks, system design, development, and testing. 

\paragraph{Objective n°2 Dataset according to the baseline}
\label{intro:objective2}

The second objective is constructing a coherent dataset with the project's baseline. The dataset contains the target variable, features, and necessary pre-processing steps. This dataset will help represent and understand the problem.

\paragraph{Objective n°3 Model for sound source localization and distance estimation}
\label{intro:objective3}

The third objective wants to create a neural network model to detect the origin of a sound using a microphone array. The neural network uses the dataset created in objective 2 for training and can localize the sound source accurately. An evaluation of the trained neural network model in a real environment allows us to see how it performs. The model will also be evaluated to understand how it can be improved.

\paragraph{Objective n°4 Realization of a dataset in a simulation}
\label{intro:objective4}

The fourth objective is to create a dataset in a simulation. The dataset will augment the real-world dataset and help provide a better neural network model. The dataset should be realistic and contain the same characteristics as the real-world dataset.

\paragraph{Objective n°5 Attacking the model to understand how it reacts}
\label{intro:objective5}

The fifth objective is to attack the model. The trained neural network model needs to be evaluated by testing it on data modified in some way, such as by adding or removing noise. Tests against various attacks, such as adversarial attacks, should be performed to understand how the model reacts.

\begin{itemize}
    \item \textbf{Chapter \ref{ch:background}: Background and Litterature}: This chapter provides an overview of the background knowledge necessary to understand the project, such as the theory of machine learning and sound propagation, and a brief introduction to sound source localization.
    \item \textbf{Chapter \ref{ch:methodology}: Design}: This chapter describes the design of the different parts of the project. It presents the description of the project's architecture and components.
    \item \textbf{Chapter \ref{ch:setup}: Realization}: This chapter describes the work done to achieve the project's objectives, such as creating a dataset, creating a simulation, building a neural network model, and evaluating it using an appropriate metric.
    \item \textbf{Chapter \ref{ch:results}: Results and Analysis}: This chapter presents the project's results, such as the performance of the neural network model and the evaluation metrics. It also analyzes the results and discusses the findings.
    \item \textbf{Chapter \ref{ch:conclusions}: Conclusion and future work}: This chapter concludes this project by discussing the main results and summarizing the key findings. 
\end{itemize}

\section{}

\newpage{\pagestyle{empty} \cleardoublepage}


\chapter{Analysis and Litterature}
\label{ch:background}

This chapter introduces technical concepts and background used in the conceptualized solution of the thesis. It also explains the analysis of the needs of the thesis and finds relations with the current state of research in sound source localization systems.

\section{Baseline analysis}
\label{sec:baseline_analysis}

During the first weeks of the thesis, we had the opportunity to place an installation of microphones on the HEIA-FR main building roof. We took that opportunity to design the baseline and analyze how to build a system around it.

After analyzing the road in front of the HEIA-FR main building, we decided to use the baseline to detect the position of vehicles driving on the road. The road is moderately busy, and the vehicles drive at a reasonable speed. The road is also straight, which makes it easier to detect the position of the vehicles. The baseline is shown in Figure \ref{fig:baseline_setup}. This analysis helps provide an intuitive understanding of the sound source localization system. The baseline comprises a vehicle as the sound source we want to record, multiple microphones recording the sound of the street, and an embedded system that manages the microphones.

\subsection{Audio file format}
\label{subsec:audio_file_format}

The Waveform Audio File Format\footnote{https://www.loc.gov/preservation/digital/formats/fdd/fdd000001.shtml} is a standard for storing numerical audio data. The audio signal is continuous and is sampled to store it on a computer. The sampling rate represents the number of samples per second. The sampling rate is usually measured in Hertz (Hz). The numerical representation of the audio often used with wav is the PCM.

The PCM (pulse-code modulation)\cite{5059525} is a method used to represent an analog signal with a binary. It is often used to represent uncompressed digital audio. The PCM is a sequence of amplitude values. Depending on the quality, the amplitude values are often stored as 8-bit, 16-bit, or 24-bit integers. A representation of a sampled sinusoidal signal is shown in Figure \ref{fig:sine_wave_representation_audacity}.

\begin{figure}[H]
    \centering
    \includegraphics[width=0.5\textwidth]{../Images/sine_wave_representation_audacity.png}
    \caption{PCM representation of a sinusoidal signal}
    \label{fig:sine_wave_representation_audacity}
\end{figure}

Each vertical line represents a sample. The horizontal axis represents the time, and the vertical axis represents the amplitude. The sampling rate is usually 44100 Hz, 48000 Hz, or 96000 Hz.

\section{Sound Source Localization}

Sound Source Localization (SSL) is the process of determining the position of a sound source. It usually uses a microphone array that captures the sound signals from multiple directions. Various applications use SSL \cite{Grumiaux_2022}, such as speech recognition \cite{7952261}, source separation \cite{8903121}, human-robot interaction \cite{Li_2016} or room acoustic analysis \cite{amengual}. In this thesis, SSL is used to estimate the distance and direction of a sound source to detect excessively noisy vehicles.

\subsection{Spectrograms for sound visualization}
\label{subsec:spectrograms}

Spectrograms are a visual representation of the frequency content of a sound signal. They are often used in sound source localization to identify the direction of a sound source. The spectrogram is a two-dimensional representation of the frequency content of a sound signal (figure \ref*{fig:spectrogram_example}).

\begin{figure}[H]
    \centering
    \includegraphics[width=0.5\textwidth]{../Images/spectrogram-example.png}
    \caption{Spectrogram of a sound signal}
    \label{fig:spectrogram_example}
\end{figure}

Spectrograms are drawn by computing the Fourier transform of a sound signal. The most common way to compute the Fourier transform is to use the Fast Fourier Transform (FFT) algorithm \cite{5217220}.

The x-axis represents time, and the y-axis represents frequency. The intensity of the color at each point in the spectrogram represents the amplitude of the frequency component. A matrix of spectrograms allows the representation of multiple channels, such as the ones recorded by a microphone array. On that matrix, each spectrogram represents the frequency content of a single channel (figure \ref*{fig:2_channel_spectrogram_example}).

\begin{figure}[H]
    \centering
    \includegraphics[width=0.4\textwidth]{../Images/2-channel-spectrogram-example.png}
    \caption{Dual channel spectrogram matrix of a sound signal}
    \label{fig:2_channel_spectrogram_example}
\end{figure}

Looking at the frequency content of the sound signal allows us to identify the time delta of a recorded sound by using a multi-channel spectrogram. The bright spot on the spectrogram will indicate a jump in the amplitude and determine the start time of the recording of a loud sound. By comparing this time with the other channel, we can find the direction of the sound source by comparing the sound signal's time delta with the other channels' time delta (figure \ref*{fig:spectrogram_offset}).

\begin{figure}[H]
    \centering
    \includegraphics[width=0.5\textwidth]{../Images/time_delta.png}
    \caption{Spectrogram of two sound signals with time their delta}
    \label{fig:spectrogram_offset}
\end{figure}

Since we know the distance between the microphones, we can determine the direction of the sound.

\subsection{Signal estimation from Short Time Fourier Transform}

The Griffin-Lim algorithm\cite{6701851} is an iterative algorithm that uses a spectrogram to estimate the phase of a signal. The algorithm starts with a random phase and iteratively updates the phase until the spectrogram converges to the original spectrogram. This algorithm allows to reconstruct a signal from a spectrogram.

\subsection{Origin of sound using two microphones}

Admitting the following setup (figure \ref*{fig:microphones_setup}), if the time delta 1 is greater than the time delta 2 of the other channels (setup 1), the sound source is closer to microphone 2. If the time delta 1 equals the time delta 2 (setup 3), the sound source is at the same distance to both microphones. If the time delta 2 is greater than the time delta 1 (setup 2), the sound source is closer to the microphone 1. 

\begin{figure}[H]
    \centering
    \includegraphics[width=1\textwidth]{../Images/microphones_setups.png}
    \caption{Sound source localization setup}
    \label{fig:microphones_setup}
\end{figure}

As explained in \cite{Scola2010DirectionOA}, once the delay between the two microphones is known, the equation allows us to find the direction of the sound source by using trigonometric calculations. As in the figure \ref*{fig:sound-source-from-two-microphones}, considering point $M$ as the sound source and point $A$ and $B$ as microphones, the distance between the two microphones is $d$ and the time delta between the two microphones is $\Delta t$, the angle $\alpha$ can be calculated.

\begin{figure}[H]
    \centering
    \includegraphics[width=0.7\textwidth]{../Images/sound-source-from-two-microphones.png}
    \caption{Equation formalization. Original image from \cite{Scola2010DirectionOA}}
    \label{fig:sound-source-from-two-microphones}
\end{figure}

Looking at the graphic allows us to find the following equation:

\begin{equation}
    AB' = AM-B'M
\end{equation}

With Pythagorean theorem: 

\begin{equation}
    AM = \sqrt{(X_{a}-X)^2 + (Y_{a}-Y)^2}
\end{equation}
\begin{equation}
    BM = \sqrt{(X_{b}-X)^2 + (Y_{b}-Y)^2}
\end{equation}

The two microphones have the same $Y$ coordinate, so $Y_{a} = Y_{b} = Y$ and $Y_{a}-Y_{b} = 0$ and $X_{a} = -X_{B}$ The equation becomes:

\begin{equation}
    y = \pm\sqrt{\frac{AB'^2}{4} - x^2_{B} + x^2(\frac{4\cdot x^2_{B}}{AB'^2} - 1)}
\end{equation}

The only variables are y and x. The value $x_{B}$ represent the position of the microphones which are taken as reference points. The value $AB'$ will not change even if the direction varies. Considering the speed of sound $c$, the distance $AB'$ is:

\begin{equation}
    AB' = c \cdot \Delta t
\end{equation}

This equation gives us the possible positions of the sound source on a line. If we admit that the sound source is always in front of the microphones, we can eliminate the negative values of $y$ and keep only the positive values. With that solution, 

This setup shows that two microphones are enough to determine the direction of a sound source.

\section{Artificial Neural networks}

Artificial neural networks, also more simply called neural networks, are machine learning algorithms based on biological neurons used to solve various problems, including image recognition, speech recognition, and natural language processing. Neural networks learn from provided data to solve a problem without explicitly programming the solution. Many domains, like self-driving cars, facial recognition, and medical imaging, achieve state-of-the-art results using neural network models. 

A neural network needs to be trained to solve a problem. The training consists of providing the neural network with examples to recognize patterns in the data. Once we finish the training, the model can solve the problem.

A neural network (figure \ref*{fig:neural_network}) is composed of multiple neurons (the circles) that are organized in layers and connected to the neurons in the previous and next layers. 

\begin{figure}[H]
    \centering
    \includegraphics[width=0.7\textwidth]{../Images/neural_network_example.png}
    \caption{Neural network}
    \label{fig:neural_network}
\end{figure}

There are two main phases in the life of a neural network: training and inference. During the training phase, the neural network is trained on a large dataset. The neural network learns to recognize patterns in the train data and tries to generalize them. During the inference phase, the neural network is used to classify new images.

Neural networks comprise multiple neurons. Neurons are mathematical functions with activation functions and weights. These determine how the neurons respond to inputs and connect to other neurons. Neural networks train themselves by adjusting the weights to minimize the error between the predicted and desired outputs. They use an optimization algorithm to adjust the neurons' weights \cite{sun2019optimization}. Multiple optimization algorithms exist, the most used is gradient descent\cite{zhang2019gradient}. These algorithms minimize the error between the predicted and desired outputs.

\subsection{Neuron}

A neuron is a mathematical function that takes multiple inputs and produces an output. The activation function and the weights of the neuron determine the output. The activation function determines how the neuron responds to inputs. The weights determine the importance of the inputs. The neuron's output is calculated by multiplying the inputs by their weights and applying the activation function to the result. The neuron sends its output to the next layer of neurons.

\paragraph{Activation function}

Multiple activation functions exist for neural networks. The most common activation functions are the linear function, the sigmoid function, the tanh function, and the ReLU function (Figure \ref{fig:activation_functions}).

\begin{figure}[H]
    \centering
    \includegraphics[width=0.7\textwidth]{../Images/activation_functions.png}
    \caption{Activation functions}
    \label{fig:activation_functions}
\end{figure}

These functions will determine how the neurons respond to inputs. These functions have been surveyed in [Activation Functions in Deep Learning: A Comprehensive Survey and Benchmark]\cite{dubey2022activation}. It shows that the ReLU function is the most used activation function in deep learning. The ReLU function is defined as:

\begin{equation}
    f(x) = max(0, x)
\end{equation}

The ReLU function is used in most neural networks because it is fast to compute and provides good results.

\paragraph{Loss function}

Neural networks use a loss function to measure the error between the predicted and desired outputs. The loss function is a mathematical function that takes the predicted and desired outputs as inputs and outputs a value representing the error between the predicted and desired outputs. The optimization algorithms use the loss function to adjust the neurons' weights to minimize the error between the predicted and desired outputs. When the loss function's value is low, the neural network accurately predicts the desired outputs.


There is a wide variety of loss functions, including the mean squared error and the cross-entropy.

The mean squared error loss function is defined as follows:

\begin{equation}
    L(y, \hat{y}) = \frac{1}{n} \sum_{i=1}^{n} (y_{i} - \hat{y}_{i})^2
\end{equation}

Where $y$ is the desired output, $\hat{y}$ is the predicted output, and $n$ is the number of classes.

The cross-entropy loss function is defined as follows:

\begin{equation}
    L(y, \hat{y}) = - \sum_{i=1}^{n} y_{i} \cdot log(\hat{y}_{i})
\end{equation}

Where $y$ is the desired output, $\hat{y}$ is the predicted output, and $n$ is the number of classes.

We can multiply the loss function by the weights to give more importance to some classes. It helps when a dataset is unbalanced by giving more importance to the underrepresented classes. The weighted cross-entropy loss function is defined as follows:

\begin{equation}
    L(y, \hat{y}) = - \sum_{i=1}^{n} w_{i} \cdot y_{i} \cdot log(\hat{y}_{i})
\end{equation}

Where $y$ is the desired output, $\hat{y}$ is the predicted output, $n$ is the number of classes, and $w$ is the weight of the class.

\paragraph{Gradient descent}

Gradient descent is an optimization algorithm that minimizes the error between the predicted and desired outputs \cite{zhang2019gradient}. We use it to train neural networks. Gradient descent works by iteratively adjusting the neurons' weights to minimize the error between the predicted and desired outputs. It uses the gradient of the loss function to find the direction of the steepest descent. The weights are adjusted in the opposite direction of the gradient. The gradient descent algorithm is defined as follows:

\begin{equation}
    \theta_{n+1} = \theta_{n} - \alpha \cdot \nabla f(\theta_{n})
\end{equation}

Where $\theta_{n}$ is the current weight, $\alpha$ is the learning rate, and $\nabla f(\theta_{n})$ is the gradient of the loss function. This algorithm is repeated until the loss function's value is low enough. The gradient descent algorithm is slow because it uses the entire dataset to compute the gradient of the loss function.

\paragraph{Mini-batch gradient descent}

Mini-batch gradient descent is a variant of gradient descent \cite{8264077}. It uses a batch of samples to compute the gradient of the loss function. It is faster than full gradient descent because it uses a mini-batch of samples instead of the entire dataset. It is also more stable than gradient descent because it uses a mini-batch of samples instead of a single sample. The mini-batch gradient descent algorithm is defined as follows:

\paragraph{Backpropagation}

Backpropagation is algorithm used to train neural networks\cite{Sekhar}. Its utility is to compute the gradient of the loss function.

\subsection{Neural network hyperparameters}
Neural networks use different hyperparameters to control the training process. Some of the most common hyperparameters are the activation function, the learning rate, the number and type of layers, and the optimizer. We use these parameters to train the neural network efficiently. Multiple solutions exist to find the best hyperparameters for a neural network, but none are perfect \cite{yu2020hyperparameter}.

\paragraph{Learning rate}

To train efficiently a neural network, we use a learning rate. It determines a factor of how much the weights are adjusted during training. A high learning rate will adjust the weights by a large amount, hence training the neural network faster but less precisely. A low learning rate will adjust the weights by a small amount, hence training the neural network slower but more precisely. The learning rate is a hyperparameter that needs to be tuned to achieve the best results.

A solution to this problem is to use an adaptive learning rate. Adaptive learning rates are learning rates that change during training. They are used to train the neural network faster and more precisely. [Adaptive Learning Rate and Momentum for Training Deep Neural Networks]\cite{hao2021adaptive} shows us how it can train neural networks efficiently without losing precision. 

\subsection{Deep Neural Networks}

Deep neural networks are a type of neural network composed of multiple layers of neurons\cite{Schmidhuber_2015}. They are trained on a large dataset and are then used to classify new data. There are countless architectures \cite{LIU201711} and implementations of neural networks, but they all share the same basic principles. The most known architectures of neural networks include CNNs\cite{oshea2015introduction}, transformers\cite{vaswani2017attention}, and many others.

\subsection{Convolutional Neural Networks for sound source localization}
\label{sec:cnn_for_ssl}

Convolutional Neural Networks (CNNs)\cite{oshea2015introduction} are deep neural networks specifically used for image recognition. They often comprise convolutional, subsampling, and fully connected layers (Figure \ref*{fig:cnn_example}).
\begin{itemize}
    \item{} Convolutional layers are used to extract features from images. These features are then fed into fully connected layers to perform classification. Each convolutional layer comprises multiple filters convolved with the input image to produce a feature map. The model trains the filters to extract specific features from the input image.
    \item{} Subsampling layers are used to reduce the size of the feature maps. The most common subsampling layer is the max-pooling layer, which takes the maximum value of a specific region of the feature map.
    \item{} Fully connected layers are trained to classify the features extracted by the convolutional layers. The output of the fully connected layers is a probability distribution over the possible classes. 
\end{itemize}

\begin{figure}[H]
    \centering
    \includegraphics[width=0.7\textwidth]{../Images/cnn_example.png}
    \caption{CNN architecture example with LeNet-5 \protect\cite{726791} composed of two convolutional layers, two subsampling layers, and finishing with two fully connected layers.}
    \label{fig:cnn_example}
\end{figure}

Even if CNNs are mainly used to classify real-life photography, they can classify any image, including sounds. The report [A survey of sound source localization with deep learning methods]\cite{Grumiaux_2022} shows that deep neural networks achieve good scores in sound source localization. CNNs can use any image as input. Based on section \ref*{subsec:spectrograms}, we can use the spectrograms as input in the network since we convert the sound into images during the spectrogram process. An approach for sound source localization is to use zones from which the sound can come as classes. The CNN will output a probability distribution over the possible classes. The class with the highest probability is the predicted class. The predicted class can then refer to a zone. The CNN then outputs a probability distribution over the possible classes. The possible classes need to be defined before training the CNN. In\cite{s20010172}, they approach the problem with 15 classes, using angles 0, 30, and 60 degrees and distances 1, 2, and 3 meters (Figure \ref*{fig:Yiwere_classes}).

\begin{figure}[H]
    \centering
    \includegraphics[width=0.5\textwidth]{../Images/Yiwere_classes.png}
    \caption{Class definition for the Yiwere classification approach\cite{s20010172}.}
    \label{fig:Yiwere_classes}
\end{figure}

This setup gives nine classes representing a different zone from where the sound can come from. The CNN will output a probability distribution over the nine classes, which allows determining the zone from where the sound comes from.

\subsection{Dropout layers}

Dropout layers help prevent overfitting in neural networks \cite{hinton2012improving}. They are stochastic techniques used to forget part of the information during training to generalize better.

\section{Datasets for machine learning}

Datasets are needed to train and test neural networks. They are composed of data and labels. The data is the neural network input, and the label is the expected output. Each entry in the dataset is composed of the data and the corresponding label. An entry in a dataset can also be called a sample. 

A balanced dataset contains nearly the same number of samples for each class. An unbalanced dataset contains a considerably different number of samples for each class.

\subsection{Datasets for sound source localization}
\label{sec:datasetsSSL}

In the case of sound source localization, the data is audio, and the labels are the zones of the sound source. 

Multiple datasets exist in sound source localization for neural networks. The most common are the DCASE 2019 task 3 dataset\cite{Adavanne2019_DCASE} and the DCASE 2020 task 3 dataset\cite{politis2020dataset}. These datasets are composed of audio files and the corresponding labels. The labels are the zones of the sound source. The audio files are recorded in a room with a microphone array and a sound source. The sound source is moved around the room, and the audio is recorded. The audio files are then annotated with the zones of the sound source. The annotations are done manually by listening to the audio files and annotating the zones. Multiple annotators then verify the annotations to ensure the quality of the annotations. 

Although these datasets are good baselines for sound source localization, they do not suit the needs of this project. The datasets are recorded in a closed environment and do not reflect the baseline defined in this project. Still, these datasets are good baselines for sound source localization and help to understand how to create a dataset.

\subsection{Dataset augmentation for audio classification}
Since recording many audio files is time-consuming and costly, and since the dataset needs to be large to train a neural network, we can use dataset augmentation techniques.

Dataset augmentation is a technique used to increase the size of a dataset. It is used to improve the performance of a neural network by training on more data, thus becoming better at generalizing. The most common techniques\cite{yang2022image} are flipping, rotating, and cropping images, but since the classification in this project is realized on audio, other techniques are needed to augment the dataset. 

Some techniques that work well on audio are adding noise, changing the pitch, or simulating new data\cite{ghilardi2019automatic}. 

\subsection{Metrics review for neural networks}

\paragraph{Accuracy}
The most common metric for neural networks is accuracy. The accuracy is the number of correct predictions divided by the total number of predictions. Accuracy is a good metric for classification problems since it tells us how the model performs. The accuracy is defined as follows:

\begin{equation}
    \text{Accuracy} = \frac{\text{Number of correct predictions}}{\text{Total number of predictions}}
\end{equation}

The accuracy is a useful metric for classification problems as it provides insight into how well the model performs. However, it may not be the best metric for unbalanced datasets. For instance, if a dataset contains 90\% of class A and 10\% of class B, a model that always predicts class A will have an accuracy of 90\%. Although this model has high accuracy, it is ineffective since it always predicts the same class.

\paragraph{Recall}
The recall is another metric that can be used. The recall is the number of true positives divided by the number of true positives and false negatives. The recall is defined as follows:

\begin{equation}
    \text{Recall} = \frac{\text{True positives}}{\text{True positives} + \text{False negatives}}
\end{equation}

The recall is a good metric for unbalanced datasets since it considers the number of false negatives.

\paragraph{F1 score}
Another metric that can be used is the F1 score. The F1 score is the harmonic mean of the precision and recall. The F1 score is defined as follows:

\begin{equation}
    \text{F1 score} = 2 \times \frac{\text{Precision} \times \text{Recall}}{\text{Precision} + \text{Recall}}
\end{equation}

\paragraph{Loss}

The loss is a metric that is calculated during the training. It can also be used to represent how well the model performs. The loss is used to update the weights of the neural network. The loss is defined as follows:

The F1 score is a good metric for unbalanced datasets since it considers precision and recall. 

\paragraph{Confusion matrix}

To have a visual understanding of how a model performs, we use a confusion matrix. A confusion matrix shows the number of correct and incorrect predictions for each class. The confusion matrix can be used with any number of classes and is a good way to visualize the performance of a model. The main aim of the visualization of a confusion matrix is to see which classes are misclassified and which classes are correctly classified. For example, in a binary classification problem, the confusion matrix is a 2x2 matrix. The confusion matrix is defined as follows:

\begin{equation}
    \begin{bmatrix}
        \text{True positive} & \text{False negative} \\
        \text{False positive} & \text{True negative}
    \end{bmatrix}
\end{equation}

And the matrix can be visualized as follows:

\begin{figure}[H]
    \centering
    \includegraphics[width=0.5\textwidth]{images/confusion_matrix_example.png}
    \caption{Confusion matrix visualization}
    \label{fig:confusion_matrix}
\end{figure}

The confusion matrix allows us to understand which classes are misclassified and to better understand if there are difficulties for the model to predict certain classes.


\section{Sound propagation}
Sound propagation is the physical process by which sound waves propagate in a given environment. Multiple factors affect the propagation of sound waves, including reverberation, occlusion, doppler effect, and obstruction.

\subsection{Microsoft Project Acoustics}


\subsubsection{Sound Propagation in game-engine}

\section{Dataset generation using simulation}
\label{sec:dataset_generation_simulation}

Simulating a dataset is a technique used to augment a dataset without recording data from real life \cite{ghilardi2019automatic}. It helps to create a dataset with many samples. 

Since this work aims to generate sounds in an open environment, a 3D-capable engine is necessary. Game engines are increasingly popular for simulation tasks since they are optimized for real-time rendering and can simulate complex 3D scenes \cite{carla}. The game engine must also simulate sound propagation to have a realistic audio representation in the simulation. 


Microsoft Project Acoustics\footnote{\url{https://learn.microsoft.com/en-us/gaming/acoustics/what-is-acoustics}} is a plugin based on Microsoft's research\cite{rosen2020interactive}. The plugin simulates sound propagation. Various applications, including video games, virtual reality, and physics simulation, use this engine. It simulates wave effects like obstruction, reverberation, and occlusion in complex 3D scenes without requiring zone markup or raytracing. It works similarly to a raytracing engine but is precomputed and optimized for real-time performance. 

 It is available in the Unity game engine. It simulates sound propagation by using ray-based acoustics methods to check for occlusion, as shown in Figure \ref{fig:microsoft_ray_based_acoustics}:

\begin{figure}[H]
    \centering
    \includegraphics[width=0.7\textwidth]{../Images/microsoft_ray_based_acoustics.png}
    \caption{Ray-based acoustics method used in Microsoft Project Acoustics to detect occlusion\cite{rosen2020interactive}.}
    \label{fig:microsoft_ray_based_acoustics}
\end{figure}

The plugin can be used in any 3D environment by pre-calculating the possible path of the sound. Once the pre-calculation is realized, the plugin can simulate the sound propagation effects in real-time.

\section{Adversarial Attacks}
\label{sec:adversarial_attacks}

Adversarial attacks are a manipulation technique that aims to fool a neural network by modifying the input data. The goal is to make the neural network misclassify. Adversarial attacks allow us to test the robustness of neural networks and understand how neural networks work and how we can improve them.

Adversarial attacks can be categorized into white-box attacks and black-box attacks. White-box attacks are attacks where the attacker has access to the neural network's parameters and architecture. Conversely, black-box attacks are attacks where the attacker cannot access the neural network's parameters and architecture.

One of the most common adversarial attacks is the Fast Gradient Sign Method (FGSM)\cite{goodfellow2015explaining}. It is a white-box attack that uses the gradient of the loss function to find the adversarial example. The adversarial example is calculated using the following equation:

\begin{equation}
    X_{adv} = X + \epsilon \cdot sign(\nabla_{X}J(\theta, X, y))
\end{equation}

$X$ is the input, $y$ is the target class, $\epsilon$ is the magnitude of the perturbation, and $J(\theta, X, y)$ is the loss function. The loss function is the function that the neural network normally tries to minimize, but here is used to maximize the loss. The gradient of the loss function is calculated for the input $X$. The sign of the gradient is then calculated and multiplied by the magnitude of the perturbation $\epsilon$. The result is added to the input to create the adversarial example $X_{adv}$. The adversarial example is then fed into the neural network, which outputs the adversarial class (Figure \ref*{fig:fgsm}).

\begin{figure}[H]
    \centering
    \includegraphics[width=0.7\textwidth]{../Images/fgsm.png}
    \caption{FGSM example in \cite{goodfellow2015explaining} with a neural network classifying a panda as a gibbon because of the attack.}
    \label{fig:fgsm}
\end{figure}

\newpage{\pagestyle{empty} \cleardoublepage}



\chapter{Methodology}
\label{ch:methodology}

This chapter presents the methodology used to create the dataset, train the neural network, and perform the adversarial attack. 

\section{Baseline design}

We must design a baseline since we start the project without previous work. The baseline is the starting point of the project. It is the simplest system that we can create to solve the problem. The baseline can then compare the results and improve the system.

The baseline system comprises three main parts: the vehicle recordings, the dataset creation, the model training, and the model testing. The system design is shown in Figure \ref{fig:baseline_system_design}.

\begin{figure}[H]
    \centering
    \includegraphics[width=1\textwidth]{../Images/baseline_system_design.drawio.png}
    \caption{Baseline system design}
    \label{fig:baseline_system_design}
\end{figure}

In Figure \ref{fig:baseline_system_design}, the red zone represents the vehicle recordings with multiple microphones and the transformation of the sound into spectrograms. We also record a video to have a ground truth to annotate the dataset. The purple zone represents the dataset creation with the spectrograms as data and the annotations from humans watching the videos as labels. We then split the dataset into a train and a test set. In the green zone, we feed the train set into a neural network to train it to predict the position of the sound source based on the spectrograms. We then test the model on the test set to evaluate its performance with unseen data.

Once we train and evaluate the model, we can use it to predict the position of a sound source based on a new recording. The model can be used in inference to detect the position of a sound source. The inference is shown in Figure \ref{fig:baseline_inference}.

\begin{figure}[H]
    \centering
    \includegraphics[width=1\textwidth]{../Images/baseline_inference.png}
    \caption{Baseline inference}
    \label{fig:baseline_inference}
\end{figure}

The inference is similar to the training and testing, except that we do not have the ground truth since our model makes the prediction. We only have to do the spectrograms from the recording, and we can feed the spectrograms into the model. The model will then predict the position of the sound source.

We can further develop this idea to incorporate other sound sources for movement tracking in a generalized environment, such as emergency vehicle detection. The baseline is a valuable starting point to develop and test a system that can accurately identify and track sound sources.

\subsection{Vehicle recordings}
\label{sec:vehicle_recordings}
To create the dataset, we must have vehicle recordings with multiple microphones. We place de microphones on the side of the street as shown in Figure \ref{fig:baseline_setup}.

\begin{figure}[H]
    \centering
    \subfloat[\centering Side view]{{\includegraphics[width=5cm]{../Images/setup_side_view.drawio.png} }}%
    \qquad
    \subfloat[\centering Top down view]{{\includegraphics[width=5cm]{../Images/setup_top_down_view.drawio.png} }}%
    \caption{Setup of the baseline}
    \label{fig:baseline_setup}
\end{figure}

We also need to record a video of the vehicle to have a ground truth to annotate the dataset. Vehicle recordings are the most crucial part of the baseline. We need to design a system that will allow us to record vehicles from the street and save the data. We designed the system managing the data recording and storage with two microphones, a camera, an embedded system, and a server to store the recordings. This system is shown in Figure \ref{fig:recording_system_design.drawio}.

\begin{figure}[H]
    \centering
    \includegraphics[width=.7\textwidth]{../Images/recording_system_design.drawio.png}
    \caption{Recording system design}
    \label{fig:recording_system_design.drawio}
\end{figure}

Overall, the baseline provides a context to develop further a concept of an accurate sound source localization system for outdoor space.

\subsection{Dataset conception}
\label{sec:dataset_conception}

The dataset is the most crucial part of the baseline. We can determine the dataset's characteristics based on the analysis of the section \ref{sec:datasetsSSL}. The dataset needs to contain the sound recorded by the microphone and the position of the sound source. To simplify the problem, we will use four classes as the main classification challenge in the project. The classes are the following:

\begin{itemize}
    \item  \textit{left\_to\_right}: The vehicle goes from the left to the right of the microphone.
    \item  \textit{right\_to\_left}:  The vehicle goes from the right to the left of the microphone.
    \item  \textit{no\_cars}:  No vehicles pass by the microphone.
    \item  \textit{multiple\_cars}:  Multiple vehicles pass by the microphone.
\end{itemize}

By adding a camera to the system in section \ref{sec:vehicle_recordings}, we can use the image captured by the camera to determine the ground truth of the sound source's position. The camera's position is the same as the microphone's position, and the camera is facing the road. These classes allow the creation of a dataset without precisely recording the vehicle's position. The \textit{no\_cars} and \textit{multiple\_cars} are here to ensure we will have a complete dataset, as with these four classes, we can cover every possible scenario recorded by the microphones and don't need to cherry-pick only the recordings that match our classification system. 

We also used only two classes at the beginning of the project to ensure the concept's functionality when installing the system. These classes are the following:

\begin{itemize}
    \item  \textit{left\_to\_right}:  The vehicle goes from the left to the right of the microphone.
    \item  \textit{right\_to\_left}:  The vehicle goes from the right to the left of the microphone.
\end{itemize}

\subsubsection{Recorded data design}

% TODO add appendix
The results and comparison of this task are available in the appendix TODO

\section{Convolutional Neural Network for Sound Source Localization}

For our baseline, we use a convolutional neural network to predict the position of the sound source. We use a convolutional neural network because, based on the analysis in section \ref{sec:cnn_for_ssl}, it is the most common neural network architecture for image classification and hence for sound source localization. We can use the spectrograms as image input and the convolutional neural network to classify the spectrograms.

Our network design is composed of a feature extraction part and a classifier part. The feature extraction part is composed of convolutional layers, ReLU, and pooling layers. The classifier part is composed of fully connected layers. The feature extraction part is used to extract the features from the spectrograms, and the classifier part is used to classify the features extracted. The full architecture is shown in Figure \ref{fig:baseline_feature_extraction}.

\begin{figure}[H]
    \centering
    \includegraphics[width=.2\textwidth]{../Images/cnn_architecture_design.drawio.png}
    \caption{Baseline feature extraction}
    \label{fig:baseline_feature_extraction}
\end{figure}

The feature extraction part comprises three blocks of one convolution, one ReLU, and one Max-pooling. The convolutional layers extract the features from the spectrograms. The ReLU layers introduce non-linearity in the network. The pooling layers reduce the dimensionality of the network. The classifier part is composed of one fully connected layer. The fully connected layer classifies the features extracted by the feature extraction part. 

\section{Simulation concept design}

To improve the classification score of the baseline, we need to have more data. Multiple possibilities are available to achieve this goal. We can record more data, but it is time-consuming and expensive. We can also use a simulation to generate new data. In this project, we use a simulation to generate new recordings to add to the training dataset to achieve a better classification score on the baseline. The simulation comprises the same elements in the recording system in section \ref{sec:vehicle_recordings} except we simulate them. The simulation design is shown in Figure \ref{fig:simulation_system_design.drawio}.

\begin{figure}[H]
    \centering
    \includegraphics[width=1\textwidth]{../Images/simulation_training_design.drawio.png}
    \caption{Simulation system design for training}
    \label{fig:simulation_system_design.drawio}
\end{figure}

There are differences with the baseline system. The main one is that we generate the data in a simulation. The second is that there is no need to annotate the dataset since we know the position of the sound source in the simulation, and we can deduce the position of the sound directly from the simulation. The last one is that we add the dataset generated by the simulation to the trainset of the baseline but not to the test set. This allows us to understand better the simulation's impact on the real data classification score.

The main advantages of the simulation are that we can generate as much data as we want. We can generate data for any position of the sound source. The simulation is composed of a vehicle, a microphone, and a camera. Based on the section \ref{sec:dataset_conception}, we want to generate data for the classes \textit{left\_to\_right} and \textit{right\_to\_left}. We can generate data for the class \textit{left\_to\_right} by placing the vehicle on the microphone's left and moving it to the right. We can generate data for the class \textit{right\_to\_left} by placing the vehicle on the right of the microphone and moving it to the left. For \textit{no\_cars} class, we can generate data by not placing the vehicle in the simulation. For the \textit{multiple\_cars} class, we can generate data by placing multiple vehicles in the simulation and moving them in the same direction. 

We need to assign an engine sound to the vehicle during the simulation for the vehicle to be recorded by the microphone.

% \begin{figure}[H]
%     \centering
%     \includegraphics[width=1\textwidth]{../Images/simulation_training_design.drawio.png}
%     \caption{Simulation system design for training}
%     \label{fig:simulation_system_design.drawio}
% \end{figure}

\subsection{Genrealization aspect of the simulation}

For the simulation to best generalize and better represent real-life data, we need to add randomness to the simulation in multiple ways. 

\paragraph{Random speed}

The vehicle's speed is not constant in real life, and we need to add randomness to the vehicle's speed in the simulation. We can add randomness to the vehicle's speed by varying the speed assigned at the beginning of the simulation. 

\paragraph{Random path}

At the beginning of the simulation, we define multiple points as possible start and end points for the vehicle journey. The vehicle's path is generated by randomly choosing a start and end point. We can then generate the vehicle's path by drawing a straight from the start to the end. This method matches the real-life scenario where the straight road in front of the HEIA-FR building constrains the vehicle's path. 

\paragraph{Random starting time}

The vehicle's arrival time is not constant in real life. We add randomness to the vehicle's starting time in the simulation to match the real-life cases. We can add randomness to the vehicle's starting time by varying the time the vehicle waits at the beginning of the simulation.

\paragraph{Random engine noise}

There are many vehicles in real life, and most have different engine noises. We can reproduce this by randomly choosing an engine noise at the beginning of the simulation and playing it during the simulation. 

\paragraph{Random background noise}





\subsection{Simulation software design}

The simulation should generate audio data by playing scenarios and recording the sound generated inside it. The comportment should represent the ones analyzed in the section \ref{sec:baseline_analysis}. Based on what 

\section{Adversarial Attack conception}

\subsection{Audio reconstruction from spectrograms}

Since the adversarial example is a spectrogram, it needs to be converted back into audio to be re-recorded through the microphone. The conversion is done using the Griffin-Lim algorithm\cite{griffin1984signal}. The Griffin-Lim algorithm is an algorithm that reconstructs an audio signal from a spectrogram. It is an iterative algorithm that uses the spectrogram to estimate the phase of the audio signal. The algorithm starts with a random phase and iteratively updates the phase until the spectrogram converges to the original spectrogram. The algorithm is defined as follows:
\newpage{\pagestyle{empty} \cleardoublepage}


\chapter{Realization}
\label{ch:setup}

\section{Simulation model creation}

\subsection{Microsoft Project plugin}

\subsection{Unity plugin}

\subsection{Unity simulation creation}

The simulation is also faster than recording the data. We can generate a lot of data in a short time. The simulation is also cheaper than recording the data. We don't need to buy a vehicle and a camera to generate the data. We only need to buy a microphone and a speaker. The simulation is also more flexible than recording the data. We can change the position of the microphone and the camera easily. We can also change the vehicle's speed and the number of vehicles in the simulation. The simulation is also more reproducible than recording the data. We can generate the same data multiple times. We can also generate data for the same position of the sound source multiple times. The simulation is also more scalable than recording the data. We can generate data for any position of the sound source. We can also generate data for any number of vehicles. The simulation is also more controllable than recording the data. We can control the position of the sound source, the vehicle's speed, and the number of vehicles. The simulation is also more secure than recording the data. We can generate data without any risk of accidents. The simulation is also more accessible than recording the data. We can generate data without

\section{Dataset creation}

\subsection{Microphone installation}

To record real data that suits our baseline, we must design a system to record and save lots of data. As we had the opportunity to place it on the HEIA-FR, we decided to design a system containing an embedded system, two microphones, a camera, and an embedded system. For the hardware, we chose to ensure that the system is easily replicable and that the system is not too expensive. We used hardware available at the HEIA and at Rosas. The system design is shown in Figure \ref{fig:real_data_recording_system}.

\begin{figure}[H]
    \centering
    \includegraphics[width=1\textwidth]{../Images/real_data_recording_system.drawio.png}
    \caption{Real data recording system}
    \label{fig:real_data_recording_system}
\end{figure}

Red flow, green flow, blue flow

\subsection{Embedded system setup and access}

The setup can be easily replicated in many other environments, such as other city streets, traffic intersections, etc. This setup is a valuable starting point to develop and test a system that can accurately identify and track sound sources.

\subsection{Data transmission}

\subsection{Data recording and storage}

\section{Neural Network for Sound Source Localization}

\subsection{Dataset preparation}

\subsection{Neural Network architecture}

\subsection{Training}

\section{Adversarial Attack}

\subsection{Fast Gradient Signed Method implementation}


\newpage{\pagestyle{empty} \cleardoublepage}


\chapter{Results}
\label{ch:results}

\section{Sound Propagation Simulation}

\subsection{Dataset Augmentation}

\section{Neural Network model results}

\section{Adversarial Attack results}

\subsection{Adversarial Attack mitigation}
\newpage{\pagestyle{empty} \cleardoublepage}


\chapter{Conclusions and Future Work}
\label{ch:conclusions}

\section{Conclusions}

\subsection{Specification Fulfillment}












\newpage{\pagestyle{empty} \cleardoublepage}


\begin{appendix}
\chapter{Appendix}
\newpage
\section{Specification}
\label{appendix:specification}
\includepdf[pages=-]{../specification/specification.pdf}

\newpage{\pagestyle{empty} \cleardoublepage}
\end{appendix}

\addcontentsline{toc}{chapter}{\numberline{}List of Tables}
\listoftables

\addcontentsline{toc}{chapter}{\numberline{}List of Figures}
\listoffigures

\addcontentsline{toc}{chapter}{\numberline{}Bibliography}

\bibliography{Bibliography/libs}
\bibliographystyle{unsrt}

%END Doc
%-------------------------------------------------------

\end{document}

